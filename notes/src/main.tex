%%%%%%%%%%%%%%%%%%%%%%%%%%%%%%%%%%%%%%%%%
% Important note:
% Chapter heading images should have a 2:1 width:height ratio,
% e.g. 920px width and 460px height.
%
% The original template (the Legrand Orange Book Template) can be found here --> http://www.latextemplates.com/template/the-legrand-orange-book
%
% Original author of the Legrand Orange Book Template:
% Mathias Legrand (legrand.mathias@gmail.com) with modifications by:
% Vel (vel@latextemplates.com)
%
% Original License:
% CC BY-NC-SA 3.0 (http://creativecommons.org/licenses/by-nc-sa/3.0/)
%%%%%%%%%%%%%%%%%%%%%%%%%%%%%%%%%%%%%%%%%
 
%----------------------------------------------------------------------------------------
%	PACKAGES AND OTHER DOCUMENT CONFIGURATIONS
%----------------------------------------------------------------------------------------

\documentclass[11pt,fleqn]{book} % Default font size and left-justified equations

\usepackage[top=3cm,bottom=3cm,left=3.2cm,right=3.2cm,headsep=10pt,letterpaper]{geometry} % Page margins

\usepackage{xcolor} % Required for specifying colors by name
\definecolor{ocre}{RGB}{52,177,201} % Define the orange color used for highlighting throughout the book

% Font Settings
\usepackage{avant} % Use the Avantgarde font for headings
%\usepackage{times} % Use the Times font for headings
\usepackage{mathptmx} % Use the Adobe Times Roman as the default text font together with math symbols from the Sym­bol, Chancery and Com­puter Modern fonts

\usepackage{microtype} % Slightly tweak font spacing for aesthetics
\usepackage[utf8]{inputenc} % Required for including letters with accents
\usepackage[T1]{fontenc} % Use 8-bit encoding that has 256 glyphs

% Bibliography
\usepackage[style=alphabetic,sorting=nyt,sortcites=true,autopunct=true,babel=hyphen,hyperref=true,abbreviate=false,backref=true,backend=biber]{biblatex}
\addbibresource{bibliography.bib} % BibTeX bibliography file
\defbibheading{bibempty}{}

\input{structure} % Insert the commands.tex file which contains the majority of the structure behind the template

\begin{document}

\title{Physics II}

%----------------------------------------------------------------------------------------
%	TITLE PAGE
%----------------------------------------------------------------------------------------

\begingroup
\thispagestyle{empty}
\AddToShipoutPicture*{\put(0,0){\includegraphics[scale=1.25]{esahubble}}} % Image background
\centering
\vspace*{5cm}
\par\normalfont\fontsize{35}{35}\sffamily\selectfont
\textbf{Physics II}\\
{\LARGE github.com/mews6}\par % Book title
\vspace*{1cm}
{Jaime Torres}\par % Author name
\endgroup

%----------------------------------------------------------------------------------------
%	COPYRIGHT PAGE
%----------------------------------------------------------------------------------------

\newpage
~\vfill
\thispagestyle{empty}

%\noindent Copyright \copyright\ 2014 Andrea Hidalgo\\ % Copyright notice

\noindent \textit{First release, 2024} % Printing/edition date

%----------------------------------------------------------------------------------------
%	TABLE OF CONTENTS
%----------------------------------------------------------------------------------------

\chapterimage{head1.png} % Table of contents heading image

\pagestyle{empty} % No headers

\tableofcontents % Print the table of contents itself

%\cleardoublepage % Forces the first chapter to start on an odd page so it's on the right

\pagestyle{fancy} % Print headers again

%----------------------------------------------------------------------------------------
%	CHAPTER 1
%----------------------------------------------------------------------------------------

\chapterimage{head2.png} % Chapter heading image

\chapter{Introduction}

First of all, Welcome! I hope i can explain Physics II in a somewhat friendly way, and i hope that whatever it is you need this text for, you can succeed on it.
Sometimes these topics can feel a bit dense (because they are) and even though not the most rigurous of texts, i hope this little guide helps you.
Now, before we start with anything, there are a few things i think we should take into account:

\textbf{There's a mistake in this book! What do i do?}

Tell me what it is, just let me know and maybe even correct it yourself, i have
no reservations on making changes in case it happens to be necessary or otherwise useful.

As a little (final) side note, here's some cool people i took Physics II with, they speak spanish and might not respond, but if you can contact them (and know how to speak spanish), they might help you!
\begin{itemize}
    \item Daniel Esteban Olaya (de.olaya1318@uniandes.edu.co)
    \item Paula Giraldo Gallo (pl.giraldo@uniandes.edu.co)
\end{itemize} 

\chapter{Fundamentals}

The II in 'Physics II' is of course, a signifier of continuity, and you sometimes don't really remember the things that you 
saw one, or a few semesters ago. So before you start thinking on the concpts unique to Physics II, a few reminders might be on
course for this text. This non-comprehensive collection of topics should be a quick reminder of a few concepts. But i urge you to 
read them on your own.

\section{Newton's Laws}

Newton's laws of motion are three basic laws of classical mechanics that describe the relationship between the motion of an object and the forces acting on it.

\subsection{First Law}

A body in state of rest, or in uniform motion in a straight line
will have an overall summatory of forces equal to 0

\begin{equation}
    \sum \vec{F} = 0
\end{equation}

\subsection{Second Law}

A net force that acts over a body makes it accelerate in the same direction
as the net force. The magnitude of acceleration is directly proportional
to the magnitude of the forces acting over it.

\begin{itemize}
    \item if a net force acts over a body, this body accelerates
    \item The direction of acceleration is the same as a net force.
\end{itemize}

we can assume:
\begin{gather}
    \vec{F}_{net} = m \vec{a} \\
    \vec{F} = \frac{d \vec{p}}{dt} 
\end{gather}

\subsection{Third Law}

When two bodies interact their forces are always equal in magnitude and
opposed in direction. This can be expressed as:

\begin{equation}
    \vec{F} AB = - \vec{F} BA
\end{equation}

\textit{\textbf{Important Equations}}
\begin{gather}
    \vec{F} = m\vec{a}\\
    \vec{T} = I\vec{\alpha}
\end{gather}

\chapter{Thermodynamics}

As it is defined on Sears and Zemanzky's University Physics:

"Thermodynamics are the study of energy transformations 
where there is an intervention between mechanical energy, 
heat, and other forms of energy (...)" \cite{1}

In this section, we'll be talking about the different ways we can analyze, comprehend and manage such topics.

\section{Temperature and Heat}
Although easily interchangeable in common day language, when talking formally, Temperature and Heat
are different physical concepts, For once, heat is a form of energy transference, measured in Joules (J). 
Temperature, instead, is an associated characteristic of an object. We'll treat both as diferent things during the course

\section{Thermal Dilation}
Temperature can make objects change their size, given a drastic enough Temperature change affecting the object.
This is defined mathematically as:

\begin{gather}
    \Delta L \propto \Delta T\\
    \Delta L \propto L_0\\
    \Delta L = \alpha L_0 \Delta T
\end{gather}

In those formulas, $\alpha$ represents the thermic expansion quoefficient. with 
their units being measured $[\alpha] = \frac{1}{C^o}, \frac{1}{K}, \lor \frac{1}{F^o}$
However, this is a quoefficient that is different depending on the material that is being worked on.

\subsection{Volume}

For a volume $V = L^3$ (such as a solid square cube), we can model a derivative of the sort:
\begin{gather}
    dV = \frac{dV}{dL}\cdot dL\\
    3L^2 dL\\
    3L^2 \alpha L_0 dT \\
    3 \alpha \underbrace{L^2 L_0}_{V_0} dT
\end{gather}

Therefore, volume changes at the rate of:

\begin{gather}
    dV = 3\alpha V_0 dT
\end{gather}

$3\alpha$ is the equivallent of a linear expansion for volumetric situations.
we can simplify it as $\beta$. Therefore it is possible to write:

\begin{gather}
    dV = \beta V_0 dT
\end{gather}

\section{The limits of Temperature}

...Eventually, we would reach the apparent universal constant of:

\begin{gather}
    T \geq 273.15 C^o
\end{gather}

This will come to define the Kelvin temperature scale, where Temperature is direcly proportional
to the pressure applied by gasses at that specific temperature. This is a scale we'll eventually 
call the absolute scale of temperature, there is no possibility of an object going to a colder 
temperature than 0 Kelvin.

\section{Thermic Equilibrium}

Thermic equilibrium is a state between two objects such as they won't have any sort of energy exchange.
This will happen if and only if both objects are at the same temperature. If two objects are at a different temperature, 
they will eventually exchange energy.

If the difference of temperature between two objects is greater, then the exchange of temperature will be larger. 

\section{Calorimetry}

Calorimetry is a field of thermodynamics preoccupied with the exchange
of heat between objects. 

Heat has a mechanic equivalent that can be measured as energy. This would eventually leave us to the conclusion
that heat is by itself, a form of energy. 

\begin{gather}
    Q = Cm\Delta T
\end{gather}

In this formula, C is a constant that measures the specific heat of a material. This can be measured as
$\frac{J}{kg \cdot (K\lor C^o)}$

\paragraph*{Example}

Suppouse an object 'A' and an object 'B', with masses '$m_a = 2gr$' and '$m_b = 1 gr$' respectively, both 
golden. at two different temperatures $T_A = 20C^o$ and $T_B = 10C^o$. How will heat behave in this system?

\begin{gather}
    \sum Q = 0\\
    \Delta T_A = T_F - T_A\\
    \Delta T_B = T_F - T_B\\ 
    m_a c_a \Delta T_A + m_b c_b \Delta T_B = 0\\
    m_a c_a (T_F - T_A) + m_b c_b (T_F - T_B) = 0\\
    m_a (T_F - T_A) + m_b (T_F - T_B) = 0\\ 
    T_F = \frac{m_a T_a + m_b T_b}{(m_a + m_b)}\\
    T_F = \frac{T_a + T_b}{2}
\end{gather}


\printbibliography

\end{document}