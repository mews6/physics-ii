%%%%%%%%%%%%%%%%%%%%%%%%%%%%%%%%%%%%%%%%%
% Important note:
% Chapter heading images should have a 2:1 width:height ratio,
% e.g. 920px width and 460px height.
%
% The original template (the Legrand Orange Book Template) can be found here --> http://www.latextemplates.com/template/the-legrand-orange-book
%
% Original author of the Legrand Orange Book Template:
% Mathias Legrand (legrand.mathias@gmail.com) with modifications by:
% Vel (vel@latextemplates.com)
%
% Original License:
% CC BY-NC-SA 3.0 (http://creativecommons.org/licenses/by-nc-sa/3.0/)
%%%%%%%%%%%%%%%%%%%%%%%%%%%%%%%%%%%%%%%%%
 
%----------------------------------------------------------------------------------------
%	PACKAGES AND OTHER DOCUMENT CONFIGURATIONS
%----------------------------------------------------------------------------------------

\documentclass[11pt,fleqn]{book} % Default font size and left-justified equations

\usepackage[top=3cm,bottom=3cm,left=3.2cm,right=3.2cm,headsep=10pt,letterpaper]{geometry} % Page margins

\usepackage{xcolor} % Required for specifying colors by name
\definecolor{ocre}{RGB}{52,177,201} % Define the orange color used for highlighting throughout the book

% Font Settings
\usepackage{avant} % Use the Avantgarde font for headings
%\usepackage{times} % Use the Times font for headings
\usepackage{mathptmx} % Use the Adobe Times Roman as the default text font together with math symbols from the Sym­bol, Chancery and Com­puter Modern fonts

\usepackage{microtype} % Slightly tweak font spacing for aesthetics
\usepackage[utf8]{inputenc} % Required for including letters with accents
\usepackage[T1]{fontenc} % Use 8-bit encoding that has 256 glyphs

% Bibliography
\usepackage[style=alphabetic,sorting=nyt,sortcites=true,autopunct=true,babel=hyphen,hyperref=true,abbreviate=false,backref=true,backend=biber]{biblatex}
\addbibresource{bibliography.bib} % BibTeX bibliography file
\defbibheading{bibempty}{}

\input{structure} % Insert the commands.tex file which contains the majority of the structure behind the template

\begin{document}
\title{Physics II}

%----------------------------------------------------------------------------------------
%	TITLE PAGE
%----------------------------------------------------------------------------------------

\begingroup
\thispagestyle{empty}
\AddToShipoutPicture*{\put(0,0){\includegraphics[scale=1.25]{esahubble}}} % Image background
\centering
\vspace*{5cm}
\par\normalfont\fontsize{35}{35}\sffamily\selectfont
\textbf{Physics II}\\
{\LARGE github.com/mews6}\par % Book title
\vspace*{1cm}
{Jaime Torres}\par % Author name
\endgroup

%----------------------------------------------------------------------------------------
%	COPYRIGHT PAGE
%----------------------------------------------------------------------------------------

\newpage
~\vfill
\thispagestyle{empty}

%\noindent Copyright \copyright\ 2014 Andrea Hidalgo\\ % Copyright notice

\noindent \textit{First release, 2024} % Printing/edition date

%----------------------------------------------------------------------------------------
%	TABLE OF CONTENTS
%----------------------------------------------------------------------------------------

\chapterimage{head1.png} % Table of contents heading image

\pagestyle{empty} % No headers

\tableofcontents % Print the table of contents itself

%\cleardoublepage % Forces the first chapter to start on an odd page so it's on the right

\pagestyle{fancy} % Print headers again

%----------------------------------------------------------------------------------------
%	CHAPTER 1
%----------------------------------------------------------------------------------------

\chapterimage{head2.png} % Chapter heading image

\chapter{Introduction}

First of all, Welcome! I hope i can explain Physics II in a somewhat friendly way, and i hope that whatever it is you need this text for, you can succeed on it.
Sometimes these topics can feel a bit dense (because they are) and even though not the most rigurous of texts, i hope this little guide helps you.
Now, before we start with anything, there are a few things i think we should take into account:

\textbf{There's a mistake in this book! What do i do?}

Tell me what it is, just let me know and maybe even correct it yourself, i have
no reservations on making changes in case it happens to be necessary or otherwise useful.

As a little (final) side note, here's some cool people i took Physics II with, they speak spanish and might not respond, but if you can contact them (and know how to speak spanish), they might help you!
\begin{itemize}
    \item Daniel Esteban Olaya (de.olaya1318@uniandes.edu.co)
    \item Paula Giraldo Gallo (pl.giraldo@uniandes.edu.co)
\end{itemize} 

\chapter{Fundamentals}

The II in 'Physics II' is of course, a signifier of continuity, and you sometimes don't really remember the things that you 
saw one, or a few semesters ago. So before you start thinking on the concpts unique to Physics II, a few reminders might be on
course for this text. This non-comprehensive collection of topics should be a quick reminder of a few concepts. But i urge you to 
read them on your own.

\section{Newton's Laws}

\begin{gather}
    \vec{F} = m\vec{a}\\
    \vec{T} = I\vec{\alpha}
\end{gather}

\chapter{Thermodynamics}





\end{document}